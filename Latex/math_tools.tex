\documentclass[12pt,a4paper]{book}

\usepackage{amsmath}
\usepackage{amsfonts, amssymb, amsthm}
\usepackage{graphicx}
\usepackage[colorlinks=true, allcolors=blue]{hyperref}
\usepackage[makeroom]{cancel} % use this to cross out things


\usepackage{algorithm2e}%Use this to create nice algorithms
\DontPrintSemicolon

%	Use this to create subfigures
\usepackage{subcaption}

%	This is used to create number in mathbb
\usepackage{bbm}

%	This is to create multirows sections in tables
\usepackage{multirow}

%	You need this for \mathds{...˚}
\usepackage{dsfont}


%	Set graphics path to the subforlders files
\graphicspath{{PCC/}{CosseratRodTheory/}{Introduction/}{ShootingMethod/}{ICA/}{ColMag/}}

%	Make it possible to have multiple and indipendent files
\usepackage{subfiles}

%	Make it possible to reference labels across files
\usepackage{xr}

%	For doing cool images, better to use tikz
\usepackage{tikz}
\usepackage{tikz-3dplot} % For 3D part, using the transformations
\usetikzlibrary{arrows}
\usetikzlibrary{quotes}
\usetikzlibrary{calc}
\usetikzlibrary{intersections}
\usetikzlibrary{positioning}%above of 



%	Set the bibliography once and for all
\bibliographystyle{ieeetr}


%	This is to make latex compile the bibliography one time either in the main or in the sub file
\newcommand{\dobib}{ % Define the command
    \bibliography{../../bibliography} % Place the path relative to the subfile here
} % Solution from : https://tex.stackexchange.com/questions/107064/bibliographies-when-using-subfiles


%	This is to make latex compile the bibliography one time either in the main or in the sub file
\newcommand{\concludeFile}[1]{ % Define the command
	
	\appendix	
	\subfile{#1}
	
    \dobib
}

%	This is to make latex compile the bibliography one time either in the main or in the sub file
\newcommand{\dobibAppendix}{ % Define the command

    \dobib
}





%%%%%%%%%%%%%%%%%%%%%%%%%%%%%%%%%%%%%%%%%%%%%%%%%%%%%%%%%%%%%%%%%%%%%%%%%%%%%%%%%%%%%%%%%%%
%%%%		COMMAND TO DECLARE FUNTION OF AND FUNCTION PARAMETERS

%	Parameters of functions
\newcommand{\s}{$s$} 							%	Just s
\newcommand{\ti}{$t$} 							%	Just t
\newcommand{\fs}{_{(s)} }						%	Function of arc length
\newcommand{\ifs}{_{i\fs}}						%	i th function of arc length
\newcommand{\fx}{_{(x)} }						%	Function of x
\newcommand{\fX}{_{(X)} }						%	Function of normalized arc length
\newcommand{\ifX}{_{i\fX}}						%	i th function of normalized arc length
\newcommand{\ft}{_{(t)} }						%	Function of time
\newcommand{\ift}{_{i\ft}}						%	i th function of time
\newcommand{\fdot}{_{(\cdot)} }				%	Fucntion of dot (to leave it void)
\newcommand{\fst}{_{\left(s, t \right)} } 	%	Function of arc length and time
\newcommand{\ifst}{_{i \fst}} 					%	i th function of arc length and time
\newcommand{\fXt}{_{\left(X, t \right)} } 	%	Function of normalized arc length and time
\newcommand{\ifXt}{_{i\left(X, t \right)} } 	%	i th function of normalized arc length and time
\newcommand{\fxi}{_{\left(\xi \right)} } 		%	Function of ξ
\newcommand{\fhau}{_{h(\tau _i)}}				%	function of τ with pedix h (ICA)
\newcommand{\f}[1]{_{(#1)}}					%	function of something passed by argument

%	Some Function needed in Hamilton principle 
\newcommand{\fvoid}{\ensuremath{\textbf{f}_{(\cdot)}}} 								%	Function of dot
\newcommand{\deps}{\ensuremath{\left.\frac{d}{d\epsilon}\right|_{\epsilon = 0}}} 	%	Derivative wrt epsilon, with epsilon=0
\newcommand{\fepsdot}{\ensuremath{\textbf{f}_{\epsilon(\cdot)}}}						%	Function of dot with pedix epsilon
\newcommand{\dotfepsdot}{\ensuremath{\dot{\textbf{f}}_{\epsilon(\cdot)}}}			%	Derivative of function of dot with pedix epsilon
\newcommand{\epsdot}{\ensuremath{\epsilon(\cdot)}}									%	Dot with pedix epsilon

\newcommand{\fq}{_{(\q)} }					%	Function of q 
\newcommand{\fqdq}{(\q, \dq)}				%	Function of q and dot q
\newcommand{\fqdqddq}{(\q, \dq, \ddq)}	%	Function of q, dot q and ddot q
\newcommand{\fdqddq}{(\dq, \ddq)}			%	Function of dot q and ddot q


\newcommand{\tn}{\ensuremath{t_n}}			%	tn
\newcommand{\tnp}{\ensuremath{t_{n+1}}}	%	tn+1
\newcommand{\tnh}{\ensuremath{t_n + h}}	%	tn + h

\newcommand{\ftn}{_{\left(t_n  \right)}}		%	function of tn
\newcommand{\ftnp}{_{\left(t_{n+1}  \right)}}	%	function of tn+1
\newcommand{\ftnh}{_{\left(t_n +h  \right)}}	%	function of tn + h


\newcommand{\fchi}{\ensuremath{_{\left(\boldchi \right)}}}				%	function of bold χ
\newcommand{\fchibar}{\ensuremath{_{\left( \bar{\boldchi} \right)}}}		%	function of bold bar χ 



%%%%%%%%%%%%%%%%%%%%%%%%%%%%%%%%%%%%%%%%%%%%%%%%%%%%%%%%%%%%%%%%%%%%%%%%%%%%%%%%%%%%%%%%%%%
%%%%		GREEK LETTERS ALREADY IN BOLD AND MATHCAL

%	Commonly used greek letter
\newcommand{\boldxi}{\ensuremath{\boldsymbol{\xi}}}						%	ξ in bold
\newcommand{\boldxiX}{\ensuremath{\boldsymbol{\xi} \fX}}					%	ξ(X) in bold
\newcommand{\boldxis}{\ensuremath{\boldsymbol{\xi} \fs}}					%	ξ(s) in bold
\newcommand{\boldxist}{\ensuremath{\boldsymbol{\xi} \fst}}				%	ξ(s, t) in bold
\newcommand{\boldxiXt}{\ensuremath{\boldsymbol{\xi} \fXt}}				%	ξ(X, t) in bold
\newcommand{\boldeta}{\ensuremath{\boldsymbol{\eta}}}						%	η in bold
\newcommand{\boldetas}{\ensuremath{\boldsymbol{\eta} \fs}}				%	η(s) in bold
\newcommand{\boldetast}{\ensuremath{\boldsymbol{\eta} \fst}}				%	η(s, t) in bold
\newcommand{\boldetaXt}{\ensuremath{\boldsymbol{\eta} \fXt}}				%	η(X, t) in bold
\newcommand{\boldLambda}{\ensuremath{\boldsymbol{\Lambda}}}				%	Λ in bold
\newcommand{\boldLambdas}{\ensuremath{\boldsymbol{\Lambda} \fs}}			%	Λ(s) in bold
\newcommand{\boldLambdast}{\ensuremath{\boldsymbol{\Lambda} \fst}}		%	Λ(s, t) in bold
\newcommand{\boldLambdaXt}{\ensuremath{\boldsymbol{\Lambda} \fXt}}		%	Λ(X, t) in bold
\newcommand{\boldLambdaa}{\ensuremath{{\boldsymbol{\Lambda}_a}}}			%	Λa in bold
\newcommand{\boldLambdaaX}{\ensuremath{{\boldsymbol{\Lambda}_a} \fX}}	%	Λa(X) in bold
\newcommand{\boldLambdaaXt}{\ensuremath{{\boldsymbol{\Lambda}_a} \fXt}}	%	Λa(X, t) in bold
\newcommand{\boldLambdaadX}{\ensuremath{{\boldLambda_{ad}} \fX}}			%	Λad(X) in bold
\newcommand{\boldLambdaadXt}{\ensuremath{{\boldLambda_{ad}} \fXt}}		%	Λad(X, t) in bold
\newcommand{\boldSigma}{\ensuremath{\boldsymbol{\Sigma}}}				%	Σ in bold
\newcommand{\boldSigmas}{\ensuremath{\boldsymbol{\Sigma} \fs}}			%	Σ(s) in bold
\newcommand{\boldSigmast}{\ensuremath{\boldsymbol{\Sigma} \fst}}			%	Σ(s, t) in bold

\newcommand{\boldepsilon}{\ensuremath{\boldsymbol{\varepsilon}}}		%	ε in bold
\newcommand{\boldkappa}{\ensuremath{\boldsymbol{\kappa}}}			%	κ in bold
\newcommand{\boldsigma}{\ensuremath{\boldsymbol{\sigma}}}			%	σ in bold
\newcommand{\boldchi}{\ensuremath{\boldsymbol{\chi}}}					%	χ in bold

%	With their derivatives
\newcommand{\dboldxi}{\ensuremath{\dot{\boldsymbol{\xi}}}}				%	dot ξ in bold
\newcommand{\dboldxis}{\ensuremath{\dot{\boldsymbol{\xi}} \fs}}			%	dot ξ(s) in bold
\newcommand{\dboldxist}{\ensuremath{\dot{\boldsymbol{\xi}} \fst}}		%	dot ξ(s, t) in bold
\newcommand{\dboldxiXt}{\ensuremath{\dot{\boldsymbol{\xi}} \fXt}}		%	dot ξ(X, t) in bold
\newcommand{\dboldeta}{\ensuremath{\dot{\boldsymbol{\eta}}}}				%	dot η in bold
\newcommand{\dboldetas}{\ensuremath{\dot{\boldsymbol{\eta}} \fs}}		%	dot η(s) in bold
\newcommand{\dboldetast}{\ensuremath{\dot{\boldsymbol{\eta}} \fst}}		%	dot η(s, t) in bold
\newcommand{\dboldetaXt}{\ensuremath{\dot{\boldsymbol{\eta}} \fXt}}		%	dot η(X, t) in bold
\newcommand{\dboldSigma}{\ensuremath{\dot{\boldsymbol{\Sigma}}}}			%	dot Σ in bold
\newcommand{\dboldSigmas}{\ensuremath{\dot{\boldsymbol{\Sigma}} \fs}}	%	dot Σ(s) in bold
\newcommand{\dboldSigmast}{\ensuremath{\dot{\boldsymbol{\Sigma}} \fst}}	%	dot Σ(s, t) in bold


\newcommand{\dboldepsilon}{\ensuremath{\dot{\boldsymbol{\varepsilon}}}}	%	dot ε in bold
\newcommand{\dboldkappa}{\ensuremath{\dot{\boldsymbol{\kappa}}}}			%	dot κ in bold
\newcommand{\dboldsigma}{\ensuremath{\dot{\boldsymbol{\sigma}}}}			%	dot σ in bold
\newcommand{\dboldchi}{\ensuremath{\dot{\boldsymbol{\chi}}}}				%	dot χ in bold

%	And double derivatives
\newcommand{\ddboldxi}{\ensuremath{\ddot{\boldsymbol{\xi}}}}				%	ddot ξ in bold
\newcommand{\ddboldxis}{\ensuremath{\ddot{\boldsymbol{\xi}} \fs}}		%	ddot ξ(s) in bold
\newcommand{\ddboldxist}{\ensuremath{\ddot{\boldsymbol{\xi}} \fst}}		%	ddot ξ(s, t) in bold
\newcommand{\ddboldxiXt}{\ensuremath{\ddot{\boldsymbol{\xi}} \fXt}}		%	ddot ξ(s, t) in bold
\newcommand{\ddboldeta}{\ensuremath{\ddot{\boldsymbol{\eta}}}}			%	ddot η in bold
\newcommand{\ddboldetas}{\ensuremath{\ddot{\boldsymbol{\eta}} \fs}}		%	ddot η(s) in bold
\newcommand{\ddboldetast}{\ensuremath{\ddot{\boldsymbol{\eta}} \fst}}	%	ddot η(s, t) in bold
\newcommand{\ddboldetaXt}{\ensuremath{\ddot{\boldsymbol{\eta}} \fXt}}	%	ddot η(s, t) in bold

%	In our work we use the constrained, allowed and zero strains
\newcommand{\boldxia}{\ensuremath{\boldsymbol{\xi}_a}}				%	ξa in bold
\newcommand{\boldxiaz}{\ensuremath{\boldsymbol{\xi}_{a_0}}}			%	ξa in bold
\newcommand{\boldxiz}{\ensuremath{\boldsymbol{\xi}_0}}				%	ξ0 in bold
\newcommand{\boldxic}{\ensuremath{\boldsymbol{\xi}_c}}				%	ξc in bold
\newcommand{\boldxias}{\ensuremath{{\boldsymbol{\xi}_a} \fs}}		%	ξa(s) in bold
\newcommand{\boldxiaX}{\ensuremath{{\boldsymbol{\xi}_a} \fX}}		%	ξa(X) in bold
\newcommand{\boldxizs}{\ensuremath{{\boldsymbol{\xi}}^0 \fs}}		%	ξ0(s) in bold
\newcommand{\boldxics}{\ensuremath{{\boldsymbol{\xi}_c} \fs}}		%	ξc(s) in bold
\newcommand{\boldxicX}{\ensuremath{{\boldsymbol{\xi}_c} \fX}}		%	ξc(X) in bold
\newcommand{\boldxiast}{\ensuremath{{\boldsymbol{\xi}_a} \fst}}		%	ξa(s, t) in bold
\newcommand{\boldxizst}{\ensuremath{{\boldsymbol{\xi}}^0 \fst}}		%	ξ0(s, t) in bold
\newcommand{\boldxicst}{\ensuremath{{\boldsymbol{\xi}_c} \fst}}		%	ξc(s, t) in bold
\newcommand{\boldxiaXt}{\ensuremath{{\boldsymbol{\xi}_a} \fXt}}		%	ξa(X, t) in bold
\newcommand{\boldxizXt}{\ensuremath{{\boldsymbol{\xi}}^0 \fXt}}		%	ξ0(X, t) in bold
\newcommand{\boldxicXt}{\ensuremath{{\boldsymbol{\xi}_c} \fXt}}		%	ξc(X, t) in bold

\newcommand{\dboldxiaX}{\ensuremath{{\dot{\boldsymbol{\xi}}}_{a\left(X \right)}}}		%	dot ξa(X) in bold
\newcommand{\dboldxiaXt}{\ensuremath{{\dot{\boldsymbol{\xi}}}_{a\left(X, t \right)}}}	%	dot ξa(X, t) in bold

%	Components of the strain in local coordinates
\newcommand{\boldK}{\ensuremath{\textbf{K}}}			%	K in bold
\newcommand{\boldKs}{\ensuremath{\textbf{K} \fs}}		%	K(s) in bold
\newcommand{\boldKst}{\ensuremath{\textbf{K} \fst}}	%	K(s, t) in bold 
\newcommand{\boldKXt}{\ensuremath{\textbf{K} \fXt}}	%	K(X, t) in bold 

\newcommand{\boldGamma}{\ensuremath{\boldsymbol{\Gamma}}}			%	Γ in bold 
\newcommand{\boldGammas}{\ensuremath{\boldsymbol{\Gamma} \fs}}		%	Γ(s) in bold 
\newcommand{\boldGammast}{\ensuremath{\boldsymbol{\Gamma} \fst}}		%	Γ(s, t) in bold 
\newcommand{\boldGammaXt}{\ensuremath{\boldsymbol{\Gamma} \fXt}}		%	Γ(X, t) in bold 

%	Derivatives of components of the strain in local coordinates
\newcommand{\dboldK}{\ensuremath{\dot{\textbf{K}}}}			%	dot K in bold
\newcommand{\dboldKs}{\ensuremath{\dot{\textbf{K}} \fs}}		%	dot K(s) in bold
\newcommand{\dboldKst}{\ensuremath{\dot{\textbf{K}} \fst}}	%	dot K(s, t) in bold 
\newcommand{\dboldKXt}{\ensuremath{\dot{\textbf{K}} \fXt}}	%	dot K(X, t) in bold 

\newcommand{\dboldGamma}{\ensuremath{\dot{\boldsymbol{\Gamma}}}}			%	dot Γ in bold 
\newcommand{\dboldGammas}{\ensuremath{\dot{\boldsymbol{\Gamma}} \fs}}	%	dot Γ(s) in bold 
\newcommand{\dboldGammast}{\ensuremath{\dot{\boldsymbol{\Gamma}} \fst}}	%	dot Γ(s, t) in bold 
\newcommand{\dboldGammaXt}{\ensuremath{\dot{\boldsymbol{\Gamma}} \fXt}}	%	dot Γ(X, t) in bold

%	Components of the strain in reference coordinates
\newcommand{\boldk}{\ensuremath{\textbf{k}}}			%	k in bold
\newcommand{\boldks}{\ensuremath{\textbf{k} \fs}}		%	k(s) in bold
\newcommand{\boldkst}{\ensuremath{\textbf{k} \fst}}	%	k(s, t) in bold 

\newcommand{\boldgamma}{\ensuremath{\boldsymbol{\gamma}}}			%	γ in bold 
\newcommand{\boldgammas}{\ensuremath{\boldsymbol{\gamma} \fs}}		%	γ(s) in bold 
\newcommand{\boldgammast}{\ensuremath{\boldsymbol{\gamma} \fst}}		%	γ(s, t) in bold 

%	Derivatives of components of the strain in reference coordinates
\newcommand{\dboldk}{\ensuremath{\dot{\textbf{k}}}}			%	dot k in bold
\newcommand{\dboldks}{\ensuremath{\dot{\textbf{k}} \fs}}		%	dot k(s) in bold
\newcommand{\dboldkst}{\ensuremath{\dot{\textbf{k}} \fst}}	%	dot k(s, t) in bold 

\newcommand{\dboldgamma}{\ensuremath{\dot{\boldsymbol{\gamma}}}}			%	dot γ in bold 
\newcommand{\dboldgammas}{\ensuremath{\dot{\boldsymbol{\gamma}} \fs}}	%	dot γ(s) in bold 
\newcommand{\dboldgammast}{\ensuremath{\dot{\boldsymbol{\gamma}} \fst}}	%	dot γ(s, t) in bold 

%	Components of the twist in local coordinates
\newcommand{\boldOmega}{\ensuremath{\boldsymbol{\Omega}}}			%	Ω in bold
\newcommand{\boldOmegas}{\ensuremath{\boldsymbol{\Omega} \fs}}		%	Ω(s) in bold
\newcommand{\boldOmegast}{\ensuremath{\boldsymbol{\Omega} \fst}}		%	Ω(s, t) in bold 
\newcommand{\boldOmegaXt}{\ensuremath{\boldsymbol{\Omega} \fXt}}		%	Ω(s, t) in bold

\newcommand{\boldV}{\ensuremath{\textbf{V}}}				%	V in bold 
\newcommand{\boldVs}{\ensuremath{\textbf{V} \fs}}			%	V(s) in bold 
\newcommand{\boldVst}{\ensuremath{\textbf{V} \fst}}		%	V(s, t) in bold 
\newcommand{\boldVXt}{\ensuremath{\textbf{V} \fXt}}		%	V(X, t) in bold 

%	Derivatives of the twist in local coordinates
\newcommand{\dboldOmega}{\ensuremath{\dot{\boldsymbol{\Omega}} }}		%	dot Ω in bold
\newcommand{\dboldOmegas}{\ensuremath{\dot{\boldsymbol{\Omega}} \fs}}	%	dot Ω(s) in bold
\newcommand{\dboldOmegast}{\ensuremath{\dot{\boldsymbol{\Omega}} \fst}}	%	dot Ω(s, t) in bold 

\newcommand{\dboldV}{\ensuremath{\dot{\textbf{V}}}}			%	dot V in bold 
\newcommand{\dboldVs}{\ensuremath{\dot{\textbf{V}} \fs}}		%	dot V(s) in bold 
\newcommand{\dboldVst}{\ensuremath{\dot{\textbf{V}} \fst}}	%	dot V(s, t) in bold 

%	Components of the twist in reference coordinates
\newcommand{\boldomega}{\ensuremath{\boldsymbol{\omega}}}			%	ω in bold
\newcommand{\boldomegas}{\ensuremath{\boldsymbol{\omega} \fs}}		%	ω(s) in bold
\newcommand{\boldomegast}{\ensuremath{\boldsymbol{\omega} \fst}}		%	ω(s, t) in bold 

\newcommand{\boldv}{\ensuremath{\textbf{v}}}				%	v in bold 
\newcommand{\boldvs}{\ensuremath{\textbf{v} \fs}}			%	v(s) in bold 
\newcommand{\boldvst}{\ensuremath{\textbf{v} \fst}}		%	v(s, t) in bold 

%	Derivatives of the twist in reference coordinates
\newcommand{\dboldomega}{\ensuremath{\dot{\boldsymbol{\omega}}}}			%	dot ω in bold
\newcommand{\dboldomegas}{\ensuremath{\dot{\boldsymbol{\omega}} \fs}}	%	dot ω(s) in bold
\newcommand{\dboldomegast}{\ensuremath{\dot{\boldsymbol{\omega}} \fst}}	%	dot ω(s, t) in bold 

\newcommand{\dboldv}{\ensuremath{\dot{\textbf{v}}}}			%	dot v in bold 
\newcommand{\dboldvs}{\ensuremath{\dot{\textbf{v}} \fs}}		%	dot v(s) in bold 
\newcommand{\dboldvst}{\ensuremath{\dot{\textbf{v}} \fst}}	%	dot v(s, t) in bold 



\newcommand{\boldzeta}{\ensuremath{\boldsymbol{\zeta}}}		%	ζ in bold


\newcommand{\boldTheta}{\ensuremath{\boldsymbol{\Theta}}}
\newcommand{\Thetaznp}{\ensuremath{\boldTheta \zn}}
\newcommand{\Thetaznpk}{\ensuremath{\boldTheta \znp^{(k)}}}


\newcommand{\Omegazn}{\ensuremath{\boldOmega \zn}}
\newcommand{\Omegaznp}{\ensuremath{\boldOmega \znp}}
\newcommand{\dOmegazn}{\ensuremath{\dot{\boldOmega} \zn}}
\newcommand{\dOmegaznp}{\ensuremath{\dot{\boldOmega} \znp}}

\newcommand{\Omegaznpk}{\ensuremath{\boldOmega \znp ^{(k)}}}
\newcommand{\dOmegaznpk}{\ensuremath{\dot{\boldOmega} \znp ^{(k)}}}



\newcommand{\boldlambda}{\ensuremath{\boldsymbol{\lambda}}}


\newcommand{\barchinp}{\ensuremath{\bar{\boldchi}_{n+1}}}
\newcommand{\fbarchinp}{\ensuremath{_{\left(\bar{\boldchi}_{n+1}\right)}}}
\newcommand{\fbarchinptnp}{\ensuremath{_{\left(\bar{\boldchi}_{n+1}, \tnp\right)}}}


\newcommand{\boldtau}{\ensuremath{\boldsymbol{\tau}}}
\newcommand{\boldtaud}{\ensuremath{\boldtau_d}}
\newcommand{\boldtaudnp}{\ensuremath{\boldtau_{d, n+1}}}


\newcommand{\boldchibar}{\ensuremath{\bar{\boldchi}}}
\newcommand{\boldchibarn}{\ensuremath{\boldchibar_n}}
\newcommand{\boldchibarnp}{\ensuremath{\boldchibar_{n+1}}}


\newcommand{\boldbeta}{\ensuremath{\boldsymbol{\beta}}}


%%%%%%%%%%%%%%%%%%%%%%%%%%%%%%%%%%%%%%%%%%%%%%%%%%%%%%%%%%%%%%%%%%%%%%%%%%%%%%%%%%%%%%%%%%%
%%%%	COMMANDS TO BE USED TO DEFINE FRAMES, TRANSFORMATIONS AND GROUPS

%	Pedices for the Newmark and Newton-Euler integrator
\newcommand{\zn}{_{0_n}}
\newcommand{\znp}{_{0_{n+1}}}

%	Groups
\newcommand{\SE}{\ensuremath{SE \left(3 \right)}}
\newcommand{\SO}{\ensuremath{SO \left(3 \right)}}
\newcommand{\se}{\ensuremath{se \left(3 \right)}}

%	Transformations, vectors, directors and matrices of Lie group
%	Trasformetions
\newcommand{\Ts}{\ensuremath{\textbf{T}\fs}}		% T(s)
\newcommand{\Tt}{\ensuremath{\textbf{T}\ft}}		% T(t)
\newcommand{\Tst}{\ensuremath{\textbf{T}\fst}}	% T(s, t)
\newcommand{\Rs}{\ensuremath{\textbf{R}\fs}}		% R(s)
\newcommand{\Rst}{\ensuremath{\textbf{R}\fst}}  	% R(s, t)
%	Quaternions
\newcommand{\Qs}{\ensuremath{\textbf{Q}\fs}}		% Q(s)
\newcommand{\QX}{\ensuremath{\textbf{Q}\fX}}		% Q(X)
\newcommand{\Qst}{\ensuremath{\textbf{Q}\fst}}  % Q(s, t)
\newcommand{\QXt}{\ensuremath{\textbf{Q}\fXt}}  % Q(X, t)
%	Position
\newcommand{\rs}{\ensuremath{\textbf{r}\fs}}		% r(s)
\newcommand{\rX}{\ensuremath{\textbf{r}\fX}}		% r(X)
\newcommand{\rst}{\ensuremath{\textbf{r}\fst}}  % r(s, t)
\newcommand{\rXt}{\ensuremath{\textbf{r}\fXt}}  % r(X, t)
%	Dircetors (Cosserat cross section)
\newcommand{\ds}[1]{\ensuremath{{\textbf{d}_{#1}} \fs}}	 		%	Director (s)
\newcommand{\dst}[1]{\ensuremath{{\textbf{d}_{#1}} \fst}}		% 	Director (s, t)
\newcommand{\ddst}[1]{\ensuremath{\dot{\textbf{d}}_{#1\fst}}} 	%	dot Director (s)

%	Same but with the subscript i
\newcommand{\Tis}{\ensuremath{\textbf{T}\ifs}}	% Ti(s)
\newcommand{\Tist}{\ensuremath{\textbf{T}\ifst}}	% Ti(s, t)
\newcommand{\Ris}{\ensuremath{\textbf{R}\ifs}}	% Ri(s)
\newcommand{\Rist}{\ensuremath{\textbf{R}\ifst}} 	% Ri(s, t)
%	Quaternions
\newcommand{\Qis}{\ensuremath{\textbf{Q}\ifs}}	% Qi(s)
\newcommand{\Qist}{\ensuremath{\textbf{Q}\ifst}} 	% Qi(s, t)
%	Position
\newcommand{\ris}{\ensuremath{\textbf{r}\ifs}}	% ri(s)
\newcommand{\rist}{\ensuremath{\textbf{r}\ifst}} 	% ri(s, t)

%	Same but with the subscript i
\newcommand{\TiX}{\ensuremath{\textbf{T}\ifX}}	% Ti(X)
\newcommand{\TiXt}{\ensuremath{\textbf{T}\ifXt}}	% Ti(X, t)
\newcommand{\RiX}{\ensuremath{\textbf{R}\ifX}}	% Ri(X)
\newcommand{\RiXt}{\ensuremath{\textbf{R}\ifXt}} 	% Ri(X, t)
%	Quaternions
\newcommand{\QiX}{\ensuremath{\textbf{Q}\ifX}}	% Qi(X)
\newcommand{\QiXt}{\ensuremath{\textbf{Q}\ifXt}} 	% Qi(X, t)
%	Position
\newcommand{\riX}{\ensuremath{\textbf{r}\ifX}}	% ri(X)
\newcommand{\riXt}{\ensuremath{\textbf{r}\ifXt}} 	% ri(X, t)


\newcommand{\rzn}{\ensuremath{\textbf{r} \zn }}
\newcommand{\rznp}{\ensuremath{\textbf{r} \znp }}
\newcommand{\drzn}{\ensuremath{\dot{\textbf{r}} \zn}}
\newcommand{\drznp}{\ensuremath{\dot{\textbf{r}} \znp}}
\newcommand{\ddrzn}{\ensuremath{\ddot{\textbf{r}} \zn}}
\newcommand{\ddrznp}{\ensuremath{\ddot{\textbf{r}} \znp}}


\newcommand{\dznp}{\ensuremath{\textbf{d} \znp}}
\newcommand{\dznpk}{\ensuremath{\textbf{d} \znp^{(k)}}}

\newcommand{\rznpk}{\ensuremath{\textbf{r}_{0_{n+1}}^{(k)}}}
\newcommand{\drznpk}{\ensuremath{\dot{\textbf{r}}_{0_{n+1}}^{(k)}}}
\newcommand{\ddrznpk}{\ensuremath{\ddot{\textbf{r}}_{0_{n+1}}^{(k)}}}

\newcommand{\Rzn}{\ensuremath{\textbf{R} \zn}}
\newcommand{\Rznp}{\ensuremath{\textbf{R} \znp}}
\newcommand{\dRzn}{\ensuremath{\dot{\textbf{R}} \zn}}
\newcommand{\dRznp}{\ensuremath{\dot{\textbf{R}} \znp}}
\newcommand{\ddRzn}{\ensuremath{\ddot{\textbf{R}} \zn}}
\newcommand{\ddRznp}{\ensuremath{\ddot{\textbf{R}} \znp}}


%	Matrices for mapping ξa and ξc
\newcommand{\B}{\ensuremath{\textbf{B}}}			%	B
\newcommand{\Bbar}{\ensuremath{\bar{\textbf{B}}}}	%	Bbar

%	Matrices of generalized stiffness and dumping
\newcommand{\Kee}{\ensuremath{\textbf{K}_{\epsilon\epsilon}}}	%	Kee
\newcommand{\Dee}{\ensuremath{\textbf{D}_{\epsilon\epsilon}}}	%	Dee

\newcommand{\Qad}{\ensuremath{\textbf{Q}_{ad}}}			%	Qad
\newcommand{\Qadt}{\ensuremath{{\textbf{Q}_{ad}}\ft}}		%	Qad(t)	
\newcommand{\QadXt}{\ensuremath{{\textbf{Q}_{ad}}\fXt}}	%	Qad(X, t)

%	Directors of world frame
\newcommand{\eu}{\ensuremath{\textbf{e}_1} }
\newcommand{\ed}{\ensuremath{\textbf{e}_2} }
\newcommand{\et}{\ensuremath{\textbf{e}_3} }
\newcommand{\e}[1]{\ensuremath{\textbf{e}_{#1}} }

%	Derivatives
\newcommand{\dTst}{\ensuremath{\dot{\textbf{T}}\fst} }	% function of time and arc length
\newcommand{\dRst}{\ensuremath{\dot{\textbf{R}}\fst} } 	% dot function of time and arc length
\newcommand{\dQst}{\ensuremath{\dot{\textbf{Q}}\fst} } 	% dot function of time and arc length
\newcommand{\drst}{\ensuremath{\dot{\textbf{r}}\fst} } 	% dot function of time and arc length



%	Some vector spaces already defined
\newcommand{\Rthree}{\ensuremath{\mathbb{R}^3}}
\newcommand{\Rfour}{\ensuremath{\mathbb{R}^4}}
\newcommand{\Rsix}{\ensuremath{\mathbb{R}^6}}
\newcommand{\Rsixsix}{\ensuremath{\mathbb{R}^{6 \times 6}}}
\newcommand{\Rn}{\ensuremath{\mathbb{R}^n}}
\newcommand{\Rna}{\ensuremath{\mathbb{R}^{n_a}}}
\newcommand{\Rne}{\ensuremath{\mathbb{R}^{n_e}}}
\newcommand{\Rnc}{\ensuremath{\mathbb{R}^{n_c}}}
\newcommand{\Rm}{\ensuremath{\mathbb{R}^m}}
\newcommand{\Rnm}{\ensuremath{\mathbb{R}^{n \times m}}}


%	Some commonly used frames
\newcommand{\Fs}{\ensuremath{\Fcal _s}}
\newcommand{\Fw}{\ensuremath{\Fcal _w}}

%	Wrenches
\newcommand{\Ws}{\ensuremath{\textbf{W}\fs} }	%	Wrench of internal stresses for Cosserat rod

%	Adjoint transformations
\newcommand{\adeta}{\ensuremath{ad_{\boldeta}}}	%	adjoint of η in bold
\newcommand{\adxi}{\ensuremath{ad_{\boldxi}}}		%	adjoint of ξ in bold

\newcommand{\adetast}{\ensuremath{ad_{\boldetast}}}	%	adjoint of η(s, t) in bold
\newcommand{\adxist}{\ensuremath{ad_{\boldxist}}}		%	adjoint of ξ(s, t) in bold

%	Adjoint transformations derivatives
\newcommand{\dadeta}{\ensuremath{ad_{\dboldeta}}}		%	adjoint of dot η in bold
\newcommand{\dadxi}{\ensuremath{ad_{\dboldxi}}}		%	adjoint of dotξ in bold

\newcommand{\dadetast}{\ensuremath{ad_{\dboldetast}}}	%	adjoint of dot η(s, t) in bold
\newcommand{\dadxist}{\ensuremath{ad_{\dboldxist}}}	%	adjoint of dot ξ(s, t) in bold

%	Forces and moments
\newcommand{\n}{\ensuremath{\textbf{n}}}			%	n
\newcommand{\ns}{\ensuremath{\textbf{n}\fs}}		%	n(s)
\newcommand{\nst}{\ensuremath{\textbf{n}\fst}}	%	n(s, t)

\newcommand{\m}{\ensuremath{\textbf{m}}}			%	m
\newcommand{\ms}{\ensuremath{\textbf{m}\fs}}		%	m(s)
\newcommand{\mst}{\ensuremath{\textbf{m}\fst}}	%	m(s, t)

%	Forces and moments with pedix i 
\newcommand{\nis}{\ensuremath{\textbf{n}\ifs}}	%	ni(s)
\newcommand{\nist}{\ensuremath{\textbf{n}\ifst}}	%	ni(s, t)

\newcommand{\mis}{\ensuremath{\textbf{m}\ifs}}	%	mi(s)
\newcommand{\mist}{\ensuremath{\textbf{m}\ifst}}	%	mi(s, t)


%	Forces and moments
\newcommand{\nX}{\ensuremath{\textbf{n}\fX}}		%	n(X)
\newcommand{\nXt}{\ensuremath{\textbf{n}\fXt}}	%	n(X, t)

\newcommand{\mX}{\ensuremath{\textbf{m}\fX}}		%	m(X)
\newcommand{\mXt}{\ensuremath{\textbf{m}\fXt}}	%	m(X, t)

%	Forces and moments with pedix i 
\newcommand{\niX}{\ensuremath{\textbf{n}\ifX}}	%	ni(X)
\newcommand{\niXt}{\ensuremath{\textbf{n}\ifXt}}	%	ni(X, t)

\newcommand{\miX}{\ensuremath{\textbf{m}\ifX}}	%	mi(X)
\newcommand{\miXt}{\ensuremath{\textbf{m}\ifXt}}	%	mi(X, t)


%%%%%%%%%%%%%%%%%%%%%%%%%%%%%%%%%%%%%%%%%%%%%%%%%%%%%%%%%%%%%%%%%%%%%%%%%%%%%%%%%%%%%%%%%%%
%%%%		ABBREVIATIONS TO USE IN THE TEXT


\newcommand{\lm}{Levenberg-Marquardt}
\newcommand{\sg}{Stewart-Gough}
\newcommand{\ie}{\textit{i.e.}}
\newcommand{\al}{arc-length}
\newcommand{\cs}{cross-section}
\newcommand{\css}{cross-sections}
\newcommand{\wrt}{with respect to}

\newcommand{\geomst}{geometrico-static}


\newcommand{\E}{Equation }
\newcommand{\Es}{Equations }

\newcommand{\Eq}[1]{Equation \eqref{#1}}

%	Change text color
\newcommand\red[1]{\ensuremath{\textcolor{red}{\textbf{#1}}}}
\newcommand\blue[1]{\textcolor{blue}{\textbf{#1}}}
\newcommand\green[1]{\textcolor{green}{\textbf{#1}}}
\newcommand\brown[1]{\textcolor{brown}{\textbf{#1}}}




%%%%%%%%%%%%%%%%%%%%%%%%%%%%%%%%%%%%%%%%%%%%%%%%%%%%%%%%%%%%%%%%%%%%%%%%%%%%%%%%%%%%%%%%%%%
%%%%		COMMAND TO USE IN EQUATIONS TO KEEP IT COOL


%	Commands fot the adjoints transformations
\newcommand{\adxin}[1]{\ensuremath{ad_{\boldxi_#1}}}
\newcommand{\addxin}[1]{\ensuremath{ad_{\dboldxi_#1}}}
\newcommand{\Adjgk}[2]{\ensuremath{Ad_{^#1\textbf{g}_#2}}}


%	Some notation in mathematical expressions
\newcommand{\dist}[1]{\bar{\textbf{#1}}}								%	Vector that act distributed on the arc length
\newcommand{\dists}[1]{\ensuremath{\bar{\textbf{#1}}\fs}}			%	Vector(s) that act distributed on the arc length
\newcommand{\distst}[1]{\ensuremath{\bar{\textbf{#1}}\fst}}		 	%	Vector(s, t) that act distributed on the arc length
\newcommand{\distX}[1]{\ensuremath{\bar{\textbf{#1}}\fX}}			%	Vector(X) that act distributed on the arc length
\newcommand{\distXt}[1]{\ensuremath{\bar{\textbf{#1}}\fXt}}		 	%	Vector(X, t) that act distributed on the arc length


%	Same but with pedix i
\newcommand{\disti}[1]{\bar{\textbf{#1}}_i}							%	i th vector that act distributed on the arc length
\newcommand{\distis}[1]{\ensuremath{\bar{\textbf{#1}}\ifs}}		    %	i th vector(s) that act distributed on the arc length
\newcommand{\distist}[1]{\ensuremath{\bar{\textbf{#1}}\ifst}}		%	i th vector(s, t) that act distributed on the arc length
\newcommand{\distiX}[1]{\ensuremath{\bar{\textbf{#1}}\ifX}}		    %	i th vector(X) that act distributed on the arc length
\newcommand{\distiXt}[1]{\ensuremath{\bar{\textbf{#1}}\ifXt}}		%	i th vector(X, t) that act distributed on the arc length

\newcommand{\zeno}{\ensuremath{\left[ 0, 1 \right]}}	%	Interval [0 1]


\newcommand{\sinL}{\ensuremath{s \in \left[ 0, \ell \right]}}	%	Parameter s in interval 0 L 
\newcommand{\Xzeno}{\ensuremath{X \in \zeno}}						%	Parameter s in interval 0 1

%	Some integral operators
\newcommand{\intzl}{\ensuremath{\int_0^\ell}}			%	Integral in arc length 0 L
\newcommand{\intzeno}{\ensuremath{\int_0^1}}			%	Integral in  0 1
\newcommand{\inttt}{\ensuremath{\int_{t_1}^{t_2}}}	%	Integral in time t1 t2
\newcommand{\intttzl}{\ensuremath{\inttt \intzl}}		% 	Integral in both time and then arc length
\newcommand{\dxdt}{\ensuremath{\; dX \;dt}}			%	dX dt to use in integrals, already spaced
\newcommand{\dsdt}{\ensuremath{\; ds \;dt}}			%	ds dt to use in integrals, already spaced
\newcommand{\inttntnp}{\int_{\tn}^{\tnp}}          	%	integral from tn to tn+1

%	Partial derivatives of lagrangian
\newcommand{\dLdeta}{\ensuremath{\frac{\partial \Lcal}{\partial \boldeta}}}	%	δL/δη
\newcommand{\dLdxi}{\ensuremath{\frac{\partial \Lcal}{\partial \boldxi}}}	%	δL/δξ
%	Partial derivatives of lagrangian in parentheses
\newcommand{\pardLdeta}{\ensuremath{\left( \dLdeta \right)}}			%	(δL/δη)
\newcommand{\pardLdxi}{\ensuremath{\left( \dLdxi \right)}}				%	(δL/δξ)
%	Some fraction
\newcommand{\derivt}{\frac{d}{dt}}
\newcommand{\derivx}{\frac{d}{dX}}
\newcommand{\derivs}{\frac{d}{ds}}
%	Some partial derivatives
\newcommand{\partit}{\ensuremath{\frac{\partial}{\partial t}} }
\newcommand{\partis}{\ensuremath{\frac{\partial}{\partial s}} }

%	Norm of a vector
\newcommand{\norm}[1]{\ensuremath{\left\lVert#1\right\rVert} }

%	Inverse operator 
\newcommand{\inv}{^{-1}}

%	Delta t for the Newmark integrator
\newcommand{\vart}{\ensuremath{\triangle t}}		%	∆t
\newcommand{\vartd}{\ensuremath{\triangle t ^2}}	%	∆t^2

%	Vector of generalized coordinates
\newcommand{\q}{\ensuremath{\textbf{q}}}				%	q in bold
\newcommand{\qs}{\ensuremath{\textbf{q}}}				%	q(s) in bold
\newcommand{\qst}{\ensuremath{\textbf{q}}}			%	q(s, t) in bold
\newcommand{\dq}{\ensuremath{\dot{\textbf{q}}}}		%	dot q in bold
\newcommand{\dqs}{\ensuremath{\dot{\textbf{q}}}}		%	dot q(s) in bold
\newcommand{\dqst}{\ensuremath{\dot{\textbf{q}}}}		%	dot q(s, t) in bold
\newcommand{\ddq}{\ensuremath{\ddot{\textbf{q}}}}		%	ddot q in bold
\newcommand{\ddqs}{\ensuremath{\ddot{\textbf{q}}}}	%	ddot q(s) in bold
\newcommand{\ddqst}{\ensuremath{\ddot{\textbf{q}}}}	%	ddot q(s, t) in bold

\newcommand{\qe}{\ensuremath{\textbf{q}_e}}				%	qe in bold
\newcommand{\dqe}{\ensuremath{\dot{\textbf{q}}_e}}		%	dot qe in bold
\newcommand{\ddqe}{\ensuremath{\ddot{\textbf{q}}_e}}		%	ddot qe in bold

\newcommand{\qet}{\ensuremath{{\textbf{q}_e}\ft}}				%	qe(X) in bold
\newcommand{\dqet}{\ensuremath{\dot{\textbf{q}}_{e(t)}}}		%	dot qe(X) in bold
\newcommand{\ddqet}{\ensuremath{\ddot{\textbf{q}}_{e(t)}}}	%	ddot qe(X) in bold

\newcommand{\qn}{\ensuremath{\textbf{q}_n}}					%	q at n
\newcommand{\qnp}{\ensuremath{\textbf{q}_{n+1}}}				%	dot q at n
\newcommand{\dqn}{\ensuremath{\dot{\textbf{q}}_n}}			%	ddot q at n
\newcommand{\dqnp}{\ensuremath{\dot{\textbf{q}}_{n+1}}}		%	q at n+1
\newcommand{\ddqn}{\ensuremath{\ddot{\textbf{q}}_n}}			%	dot q at n+1
\newcommand{\ddqnp}{\ensuremath{\ddot{\textbf{q}}_{n+1}}}	%	ddot q at n+1

\newcommand{\qnpk}{\ensuremath{\textbf{q}_{n+1}^{(k)}}}			%	q at n+1 for iteration (k)
\newcommand{\dqnpk}{\ensuremath{\dot{\textbf{q}}_{n+1}^{(k)}}}	%	dot q at n+1 for iteration (k)
\newcommand{\ddqnpk}{\ensuremath{\ddot{\textbf{q}}_{n+1}^{(k)}}}	%	ddot q at n+1 for iteration (k)


%	Some Jacobians
\newcommand{\J}{\ensuremath{\textbf{J}}}			%	J in bold
\newcommand{\Js}{\ensuremath{\textbf{J} \fs}}		%	J(s) in bold
\newcommand{\Jst}{\ensuremath{\textbf{J} \fst}}	%	J(s, t) in bold
\newcommand{\JX}{\ensuremath{\textbf{J} \fX}}		%	J(X) in bold
\newcommand{\JXt}{\ensuremath{\textbf{J} \fXt}}	%	J(X, t) in bold

\newcommand{\dJ}{\ensuremath{\dot{\textbf{J}}}}			%	J in bold
\newcommand{\dJs}{\ensuremath{\dot{\textbf{J}} \fs}}		%	J(s) in bold
\newcommand{\dJst}{\ensuremath{\dot{\textbf{J}} \fst}}	%	J(s, t) in bold
\newcommand{\dJX}{\ensuremath{\dot{\textbf{J}} \fX}}		%	J(X) in bold
\newcommand{\dJXt}{\ensuremath{\dot{\textbf{J}} \fXst}}	%	J(X, t) in bold

%	The Psi matrix
\newcommand{\boldPhi}{\ensuremath{\boldsymbol{\Phi}}}				%	Φ in bold
\newcommand{\boldPhis}{\ensuremath{\boldsymbol{\Phi}\fs}}		%	Φ(s) in bold
\newcommand{\boldPhiX}{\ensuremath{\boldsymbol{\Phi}\fX}}		%	Φ(X) in bold
\newcommand{\boldPhist}{\ensuremath{\boldsymbol{\Phi}\fst}}		%	Φ(s, t) in bold
\newcommand{\boldPhiXt}{\ensuremath{\boldsymbol{\Phi}\fXt}}		%	Φ(X, t) in bold


%	Vectors for the coordinates of Newton-Euler
\newcommand{\Qa}{\ensuremath{\textbf{Q}_a}}
\newcommand{\Qar}{\ensuremath{\textbf{Q}_{a_r}}}
\newcommand{\Qaeps}{\ensuremath{\textbf{Q}_{a_\epsilon}}}
\newcommand{\Qadeps}{\ensuremath{\textbf{Q}_{a_{\epsilon d}}}}
\newcommand{\Qadepsn}[1]{\ensuremath{\textbf{Q}_{a_{\epsilon d, #1}}}}
\newcommand{\Qv}{\ensuremath{\textbf{Q}_v}}
\newcommand{\Qg}{\ensuremath{\textbf{Q}_g}}
\newcommand{\Qe}{\ensuremath{\textbf{Q}_e}}
\newcommand{\Qeps}{\ensuremath{\textbf{Q}_\epsilon}}
\newcommand{\Qenp}{\ensuremath{\textbf{Q}_{e, n+1}}}
\newcommand{\Qadtnp}{\ensuremath{\textbf{Q}_{ad \left(t_{n+1} \right)}}}


%	Some definitions for the residual
\newcommand{\Rcaln}{\ensuremath{\boldRcal_{n}}}
\newcommand{\Rcalqn}{\ensuremath{\boldRcal_{q, n}}}
\newcommand{\Rcallambdan}{\ensuremath{\boldRcal_{\lambda, n}}}
\newcommand{\Rcalnp}{\ensuremath{\boldRcal_{n+1}}}
\newcommand{\Rcalqnp}{\ensuremath{\boldRcal_{q, n+1}}}
\newcommand{\Rcallambdanp}{\ensuremath{\boldRcal_{\lambda, n+1}}}


%%%%%%%%%%%%%%%%%%%%%%%%%%%%%%%%%%%%%%%%%%%%%%%%%%%%%%%%%%%%%%%%%%%%%%%%%%%%%%%%%%%%%%%%%%%
%%%%		SOME VARIABLES IN BOLD AND MATHCAL

%	All in mathcal
\newcommand{\Acal}{\ensuremath{\mathcal{A}} }
\newcommand{\Ccal}{\ensuremath{\mathcal{C}} }
\newcommand{\Dcal}{\ensuremath{\mathcal{D}} }
\newcommand{\Fcal}{\ensuremath{\mathcal{F}} }
\newcommand{\Hcal}{\ensuremath{\mathcal{H}} }
\newcommand{\Jcal}{\ensuremath{\mathcal{J}} }
\newcommand{\Kcal}{\ensuremath{\mathcal{K}} }
\newcommand{\Lcal}{\ensuremath{\mathcal{L}} }
\newcommand{\Mcal}{\ensuremath{\mathcal{M}} }
\newcommand{\Rcal}{\ensuremath{\mathcal{R}} }
\newcommand{\Tcal}{\ensuremath{\mathcal{T}} }
\newcommand{\Ucal}{\ensuremath{\mathcal{U}} }
\newcommand{\Wcal}{\ensuremath{\mathcal{W}} }


%	Some with counterpart in bold
\newcommand{\boldAcal}{\ensuremath{\boldsymbol{\Acal}} }
\newcommand{\boldDcal}{\ensuremath{\boldsymbol{\Dcal}} }
\newcommand{\boldFcal}{\ensuremath{\boldsymbol{\Fcal}} }
\newcommand{\boldHcal}{\ensuremath{\boldsymbol{\Hcal}} }
\newcommand{\boldMcal}{\ensuremath{\boldsymbol{\Mcal}} }
\newcommand{\boldRcal}{\ensuremath{\boldsymbol{\Rcal}} }

\newcommand{\boldDcala}{\ensuremath{\boldsymbol{\Dcal}_a} }
\newcommand{\boldHcala}{\ensuremath{\boldsymbol{\Hcal}_a} }
\newcommand{\boldKcal}{\ensuremath{\boldsymbol{\Kcal}} }
\newcommand{\boldKcalee}{\ensuremath{\boldKcal_{\epsilon\epsilon}} }

\newcommand{\boldJcal}{\ensuremath{\boldsymbol{\Jcal}}}


%	Generalization
\newcommand{\mcal}[1]{\ensuremath{\mathcal{#1}} }
\newcommand{\bmcal}[1]{\ensuremath{\boldsymbol{\mcal{#1}}} }



%%%%%%%%%%%%%%%%%%%%%%%%%%%%%%%%%%%%%%%%%%%%%%%%%%%%%%%%%%%%%%%%%%%%%%%%%%%%%
%%%%%%%%%%%		TEMP		TEMP		TEMP		TEMP		TEMP

\newcommand{\Idof}{\ensuremath{\boldsymbol{\mathcal{I}}_{dof}}}  
\newcommand{\rangelimbs}{\forall i \in \left[1, N \right]}
\newcommand{\nlimb}{{N_{limbs}}}


\newcommand{\pluseq}{\mathrel{+}=}
\newcommand{\mineq}{\mathrel{-}=}






 





\newcommand{\Fa}{\ensuremath{\mathcal{F}_a}}
\newcommand{\Fb}{\ensuremath{\mathcal{F}_b}}
\newcommand{\Fe}{\ensuremath{\mathcal{F}_e}}

\newcommand{\Ba}{\ensuremath{\mathcal{B}_a}}
\newcommand{\Bb}{\ensuremath{\mathcal{B}_b}}

\newcommand{\agb}{\ensuremath{{^a g _b}}}
\newcommand{\aRb}{\ensuremath{{^a R _b}}}
\newcommand{\rba}{\ensuremath{{r_{b/a}}}}
\newcommand{\hatrba}{\ensuremath{{\hat{r}_{b/a}}}}

\newcommand{\etaa}{\ensuremath{{\eta_{a}}}}
\newcommand{\etab}{\ensuremath{{\eta_{b}}}}
\newcommand{\etaba}{\ensuremath{{\eta_{b/a}}}}

\newcommand{\dotetaa}{\ensuremath{{\dot{\eta}_{a}}}}
\newcommand{\dotetab}{\ensuremath{{\dot{\eta}_{b}}}}
\newcommand{\dotetaba}{\ensuremath{{\dot{\eta}_{b/a}}}}

\newcommand{\Adagb}{\ensuremath{Ad_{\agb}}}
\newcommand{\dotAdagb}{\ensuremath{\dot{Ad}_{\agb}}}


\newcommand{\adetaba}{\ensuremath{ad_{\etaba}}}


\newcommand{\Dzetaba}{\ensuremath{{\Delta \zeta_{b/a}}}}

\newcommand{\adDzetaba}{\ensuremath{ad_{\Dzetaba}}}



\newcommand{\kgj}{\ensuremath{{^k g _j}}}
\newcommand{\jgk}{\ensuremath{{^j g _k}}}

\newcommand{\Adkgj}{\ensuremath{Ad_{\kgj}}}
\renewcommand{\Adjgk}{\ensuremath{Ad_{\jgk}}}

\newcommand{\doteta}{\ensuremath{\dot{\eta}}}

\newcommand{\etaj}{\ensuremath{\eta_j}}
\newcommand{\etak}{\ensuremath{\eta_k}}
\newcommand{\etajk}{\ensuremath{\eta_{j/k}}}
\newcommand{\etakj}{\ensuremath{\eta_{k/j}}}



\tikzset{
    axis/.style={-stealth,line width=2pt, line cap=round},
    xaxis/.style={axis,red},
    yaxis/.style={axis,green},
    zaxis/.style={axis,blue},
}

\begin{document}


\begin{titlepage}
   \begin{center}
       \vspace*{1cm}

		\textbf{Package Report}		
		
		\vspace{1.5cm}
		
        \Large{\textbf{math\_tools}}

        %\vspace{0.5cm}
        %Thesis Subtitle
            
        \vspace{1.5cm}

        \textbf{Andrea Gotelli}

        \vfill
            
%       A thesis presented for the degree of\\
%       Doctor of Philosophy
            
        \vspace{0.8cm}
     
        %\includegraphics[width=0.4\textwidth]{university}
            
		\begin{flushleft}
			Laboratoire des Sciences du Numérique de Nantes LS2N
		\end{flushleft}
        École centrale de Nantes, Nantes\\
        France\\
       %Date
       
       
            
   \end{center}
  
\end{titlepage}


\tableofcontents

\listoffigures






\chapter{Lie Algebra Operations}

\tdplotsetmaincoords{70}{115}
\begin{figure}
	\centering
	\begin{tikzpicture}[tdplot_main_coords]
	
		\coordinate (Fe) at (0, 0, 0);
		
		
		\draw[xaxis] (Fe) -- (1,0,0) node[pos=1.3]{\( x \)};
    	\draw[yaxis] (Fe) -- (0,1,0) node[pos=1.3]{\( y \)};
      	\draw[zaxis] (Fe) -- (0,0,1) node[pos=1.3]{\( z \)};
      	\shade [ball color=black] (Fe) circle [radius=1.5pt];
      	\node[anchor=south east] at (Fe) {\Fe};	
      	
      	
      	
      	
      	\pgfmathsetmacro{\rvec}{3}
		\pgfmathsetmacro{\thetavec}{30}
		\pgfmathsetmacro{\phivec}{60}
		\pgfmathsetmacro{\alphavec}{-30}
		
		\tdplotsetcoord{P1}{\rvec}{\thetavec}{\phivec}
		\tdplotsetrotatedcoords{\phivec}{\thetavec}{\alphavec}
		\tdplotsetrotatedcoordsorigin{(P1)}
		\draw[xaxis,tdplot_rotated_coords] (0,0,0) -- (1,0,0) node[pos=1.3]{\( x \)};
		\draw[yaxis,tdplot_rotated_coords] (0,0,0) -- (0,1,0) node[pos=1.3]{\( y \)};
		\draw[zaxis,tdplot_rotated_coords] (0,0,0) -- (0,0,1) node[pos=1.3]{\( z \)};
		\shade [ball color=black] (P1) circle [radius=1.5pt];	
		\node[anchor=east] at (P1) {\Fa};
		
		
		\pgfmathsetmacro{\rvec}{3}
		\pgfmathsetmacro{\thetavec}{30}
		\pgfmathsetmacro{\phivec}{-60}
		\pgfmathsetmacro{\alphavec}{-30}
		
		\tdplotsetcoord{P2}{\rvec}{\thetavec}{\phivec}
		\tdplotsetrotatedcoords{\phivec}{\thetavec}{\alphavec}
		\tdplotsetrotatedcoordsorigin{(P2)}
		\draw[xaxis,tdplot_rotated_coords] (0,0,0) -- (1,0,0) node[pos=1.3]{\( x \)};
		\draw[yaxis,tdplot_rotated_coords] (0,0,0) -- (0,1,0) node[pos=1.3]{\( y \)};
		\draw[zaxis,tdplot_rotated_coords] (0,0,0) -- (0,0,1) node[pos=1.3]{\( z \)};
		\shade [ball color=black] (P2) circle [radius=1.5pt];
		\node[anchor=west] at (P2) {\Fb};
      	
      	
      	
      		
		
		
	
%		\draw[-stealth, line width=2pt,red,line cap=round] (Fa) -- (1,0,0) node[pos=1.3]{\( x \)};
%    	\draw[-stealth, line width=2pt,green,line cap=round] (Fa) -- (0,1,0) node[pos=1.3]{\( y \)};
%      	\draw[-stealth, line width=2pt,blue,line cap=round] (Fa) -- (0,0,1) node[pos=1.3]{\( z \)};
%      	\shade [ball color=black] (Fa) circle [radius=1.5pt];
%      	
%      	
%      	\coordinate (Fb) at (5, 4, 3);
%	
%		\draw[-stealth, line width=2pt,red,line cap=round] (Fb) -- (1,0,0) node[pos=1.3]{\( x \)};
%    	\draw[-stealth, line width=2pt,green,line cap=round] (Fb) -- (0,1,0) node[pos=1.3]{\( y \)};
%      	\draw[-stealth, line width=2pt,blue,line cap=round] (Fb) -- (0,0,1) node[pos=1.3]{\( z \)};
%      	\shade [ball color=black] (Fa) circle [radius=1.5pt];
	\end{tikzpicture}
	\caption{Representation of two generic frames : \Fa and \Fb in space, with the reference or Euclidean frame \Fe.}
\end{figure}


Note that we use \aRb to express the orientation of the frame \Fb with respect to the frame \Fa and \rba as the position of frame \Fb with respect to the frame \Fa.

Given the homogeneous tranformation from frame \Fa to frame \Fb, namely \agb, defined as $\agb = \langle \aRb,\; \rba \rangle$. This create a $4 \times 4$ matrix acting as a map in the Lie group.



\begin{equation}
	\agb = 
	\begin{bmatrix}
		\aRb			&	\rba	\\
		0_{[3\times1]}	&	1
	\end{bmatrix}
\end{equation}

This operator uniquely defines the pose of a frame with respect to a reference or euclidean frame, namely \Fw or \Fe.
From this definition we can define the derivatives of this operator with respect to time expressed with respect to the reference frame for the two frames.

\begin{equation}
\begin{aligned}
	\frac{d}{dt}\left( g_a \right) &= H_{a} \\
	\frac{d}{dt}\left( g_b \right) &= H_{b}
\end{aligned}
\end{equation}

These two derivatives can be expressed in local coordinates for every frame, by reprojecting the derivative into the local coordinates frame.

\begin{equation}
\begin{aligned}
	g_a \inv \frac{d}{dt}\left( g_a \right) &= \eta_{a} \\
	g_b \inv \frac{d}{dt}\left( g_b \right) &= \eta_{b}
\end{aligned}
\end{equation}

The twist $\eta_{a}$ and $\eta_{b}$ expresses the velocities of the frame \Fa{}, respectively \Fb{}, in the local coordinate frame or, "in their point of view".

If we consider the relative pose \agb{}, the derivative with respect to time gives 

\begin{equation}
	\agb \inv \frac{d}{dt}\left( \agb \right) = \eta	_{b/a}
\end{equation}

Which the twist of \Fb{} with respect to the twist of \Fa{} seen from this last one.

When we have a series of connected bodies, we will need to propagate the kinematics along the chain. When we have a twist in a body \Bb{}, namely \etab{}, expressed in the local coordinates of frame \Fb{} we then might need to propagate this twist to the body \Ba{} with relative pose from \Fb{} to \Fa{} given by \agb{}. In this case we use the adjoint operator \Adagb{} defined as follows :


\begin{equation}
	\Adagb = 
	\begin{bmatrix}
		\aRb	&	0_{[3\times 3]}	\\
		\hatrba \aRb	&	\aRb
	\end{bmatrix}
\end{equation}

We thus have 

\begin{equation}
\begin{aligned}
	\etaa 	&= \Adagb \etab
			&=
			\begin{bmatrix}
				\aRb	&	0_{[3\times 3]}	\\
				\hatrba \aRb	&	\aRb
			\end{bmatrix}
			\begin{bmatrix}
				\Omega_b	\\
				V_b
			\end{bmatrix}
\end{aligned}
\end{equation}


If we need to compute the acceleration than we need to derivate this relation. We thus have that 

\begin{equation}
\begin{aligned}
	\dotetaa 	&= \frac{d}{dt}\left( \Adagb \etab \right) \\
				&= \dotAdagb \etab + \Adagb \dotetab
\end{aligned}
\end{equation}

The derivative of the adjoint operator, namely \dotAdagb{}, is defined as follows

\begin{equation}\label{eq:derivative Ad time}
	\dotAdagb = \Adagb \adetaba
\end{equation}

Where the \adetaba operator accounts for the contribution of the twist in the derivative with respect to time; This operator is defined as :

\begin{equation}
	\adetaba = 
	\begin{bmatrix}
		\hat{\Omega}_{b/a}	&	0 \\
		\hat{V}_{b/a}			&	\hat{\Omega}_{b/a}
	\end{bmatrix}
\end{equation}


If we now want to determine the tangent kinematics we first need to introduce 

\begin{equation}\label{eq: Delta zeta}
	\Delta \zeta_{a} = g_a \inv \Delta g_a
\end{equation}


We can compute the tangent of the twist as follows

\begin{equation}
	\Delta \etaa = g_a \inv \Delta \frac{d}{dt}\left(g_a \right)
\end{equation}

We can now project this twist as usual and as we described before

\begin{equation}
\begin{aligned}
	\Delta \etaa 	&= \Delta  \left( \Adagb \etab \right)
					&= \Delta \Adagb \etab  + \Adagb \Delta \etab 
\end{aligned}
\end{equation}

Similarly to what we introduced before, we can define the operator $\Delta \Adagb$.

\begin{equation}
	\Delta \Adagb = \Adagb \adDzetaba
\end{equation}



We now have that, given a relative pose \agb{} we can compute \Dzetaba{} as 

\begin{equation}
	\Dzetaba = \Delta \zeta _b - \Delta \zeta _a 
\end{equation}


if we now consider the accelerations, we derivate the equation to obtain 

\begin{equation}
\begin{aligned}
	\Delta \dotetaa 	&= \Delta  \left( \dotAdagb \etab + \Adagb \dotetab \right) \\
					&= 
\Delta \dotAdagb \etab+
\dotAdagb \Delta \etab
+
\Delta \Adagb \dotetab+
\Adagb \Delta \dotetab							
\end{aligned}
\end{equation}

And we can define the operator $\Delta \dotAdagb$  as follows

\begin{equation}
\begin{aligned}
	\Delta \dotAdagb 	&= \Delta  \left( \Adagb \adetaba \right)  \\
		&=  \Delta \Adagb \adetaba +  \Adagb \Delta \adetaba \\
		&= \Adagb \adDzetaba \adetaba + \Adagb ad_{\Delta \etaba} \\
		&= \Adagb \left[ \adDzetaba \adetaba + \Adagb ad_{\Delta \etaba} \right]
\end{aligned}
\end{equation}



\paragraph{We want to show that $\dot{Ad}_g^{-T} = - Ad_g^{-T} ad_\eta ^T = - ad^T _{Ad_g \eta} Ad_g ^{-T}$}

We need to invoke the duality between twists and wrenches in order to find that, for every virtual twist $\delta \zeta$

\begin{equation}\label{eq:duality twists wrenches}
\begin{aligned}
	\delta \zeta ^T \left( Ad_g^{-T} ad_\eta^T F  \right) &= \delta \zeta ^T \left( ad_{Ad_g \eta}^TAd_g^{-T} F  \right) \\
	\left(  ad_\eta Ad_g^{-1} \delta \zeta  \right)^T F &= \left( Ad_g^{-1} ad_{Ad_g \eta}  \delta \zeta  \right)^T F \\
	&= \left( Ad_g^{-1} ad_{Ad_g \eta}  Ad_g Ad_g ^{-1}\delta \zeta  \right)^T F
\end{aligned}
\end{equation}

Now using the property that $ad_{Ad_g \eta}  Ad_g = Ad_g ad_{ \eta}$ \color{red} \textbf{proof required!}\color{black} we can simplify the right part of Equation \eqref{eq:duality twists wrenches} as follows

\begin{equation}\label{eq:duality twists wrenches simplified}
\begin{aligned}
	\left(  ad_\eta Ad_g^{-1} \delta \zeta  \right)^T F &= \left( Ad_g^{-1} Ad_g ad_{ \eta} Ad_g ^{-1}\delta \zeta  \right)^T F \\
	&= \left( ad_{ \eta} Ad_g ^{-1}\delta \zeta  \right)^T F
\end{aligned}
\end{equation}

As Equations \eqref{eq:duality twists wrenches} and \eqref{eq:duality twists wrenches simplified} are valied for every wrench $F$ and twist $\delta \zeta$, we have proven the realation

\begin{equation}
	\dot{Ad}_g^{-T} = - Ad_g^{-T} ad_\eta ^T = - ad^T _{Ad_g \eta} Ad_g ^{-T}
\end{equation}


\subsection{Definition of relative acceleration}

Having two bodies namely $\beta_k$ and $\beta_j$, with $j>k$, we can define the relative twist as $\eta_{j/k}$ where in the pedix $j/k$ the $/$ means with respect to; in this case it expressed the twist $\eta_j$ with respect to the frame attached to the body $\beta_k$.

If we know the twist of the body $\beta_k$, expressed in its local coordinates, as $\eta_k$, then we can compute the twist of body $\beta_j$ expresses in the local coordinated frame attached to the body $\beta_j$ as 

\begin{equation}\label{eq: twist definition}
	\eta_j = Ad_{^j g_k} \eta_k + \eta_{j/k}
\end{equation}

We can differentiate this equation in order to obtain 
\begin{equation}\label{eq:twist derivative raw}
\begin{aligned}
	\dot{\eta}_j 	&= Ad_{^jg_k} \dot{\eta}_k + \dot{Ad}_{^jg_k} \eta_k + \dot{\eta}_{j/k} \\
					&= Ad_{^jg_k} \dot{\eta}_k + Ad_{^jg_k} ad_{\eta_{k/j}} \eta_k + \dot{\eta}_{j/k} 
\end{aligned}
\end{equation}

Equation \eqref{eq:twist derivative raw} needs the knowledge of the relative twist $\eta_{k/j}$ expressing the twist of the body $k$ as seen in the coordinate frame attached to body $j$. In our Newton-Euler formalism, we have the opposite quantity $\eta_{j/k}$. We thus need to find an expression of Equation \eqref{eq:twist derivative raw} that uses $\eta_{j/k}$ instead of $\eta_{k/j}$.


We start from the identity ${^jg_k}{^kg_j} = \mathbb{1}$. If we differentiate it with respect to time we obtain 

\begin{equation}\label{eq: derivative equality poses}
\begin{aligned}
	{^j\dot{g}_k}{^kg_j} 	&= - {^jg_k}{^k\dot{g}_j} \\
							&= - {^j} \hat{\eta}_{j/k}
\end{aligned}
\end{equation}


By multiplying the left hand side of Equation \eqref{eq: derivative equality poses} by ${^jg_k}{^kg_j}$ we obtain


\begin{equation}\label{eq: derivative equality poses 2}
\begin{aligned}
	{^jg_k}{^kg_j}{^j\dot{g}_k}{^kg_j} 	&= - {^j} \hat{\eta}_{j/k}\\
	{^jg_k} \hat{\eta}_{k/j}{^kg_j}		&= 
\end{aligned}
\end{equation}

Using the relation $\left[ g\eta g \inv \right]^\vee = Ad_g \eta $ \color{red}\textbf{proof required!}\color{black} in the left hand side of Equation \eqref{eq: derivative equality poses 2} we obtain the following expression.

\begin{equation}\label{eq: derivative equality poses 3}
\begin{aligned}
	{^jg_k}{^kg_j}{^j\dot{g}_k}{^kg_j} 	&= - {^j \hat{\eta}}_{j/k}\\
	Ad_{^jg_k} \eta_{k/j}					&= - {^j {\eta}}_{j/k} \\
	\eta_{k/j} &= - Ad_{^jg_k} \inv {^j {\eta}}_{j/k}
\end{aligned}
\end{equation}

From the right hand side of Equation \eqref{eq: derivative equality poses 3} we can apply the relation $Ad_g \inv = Ad_{} g\inv$ in order to obtain

\begin{equation}\label{eq: derivative equality poses 4}
\begin{aligned}
	\eta_{k/j} 	&= - Ad_{^jg_k} \inv {^j {\eta}}_{j/k} \\
				&= - Ad_{^jg_k\inv}  {^j {\eta}}_{j/k} \\
				&= - Ad_{^kg_j}  {^j {\eta}}_{j/k} \\
\end{aligned}
\end{equation}

Substituing Equation \eqref{eq: derivative equality poses 4} into Equation \eqref{eq:twist derivative raw} we obtain the following expression.

\begin{equation}\label{eq:twist derivative 1}
	\dot{\eta}_j = Ad_{^jg_k} \dot{\eta}_k + Ad_{^jg_k} ad_{- Ad_{^kg_j} {^j {\eta}}_{j/k}} \eta_k + \dot{\eta}_{j/k} 
\end{equation}

Introducing the relation $Ad_g Ad_g\inv = \mathbb{1}$ in Equation \eqref{eq:twist derivative 1} we obtain

\begin{equation}\label{eq:twist derivative 2}
\begin{aligned}
	\dot{\eta}_j 	&= Ad_{^jg_k} \dot{\eta}_k - Ad_{^jg_k} ad_{Ad_{^kg_j} {^j {\eta}}_{j/k}} Ad_{^kg_j} Ad_{^kg_j} \inv \eta_k + \dot{\eta}_{j/k} \\ 
					&= Ad_{^jg_k} \dot{\eta}_k - Ad_{^jg_k} ad_{Ad_{^kg_j} {^j {\eta}}_{j/k}} Ad_{^kg_j} Ad_{^jg_k} \eta_k + \dot{\eta}_{j/k} 
\end{aligned}
\end{equation}


\paragraph{Lemma 1}
We want to show that 
\begin{equation}
	ad_{Ad_g \eta_1} Ad_g \eta_2 = Ad_g ad_{\eta_1} \eta_2
\end{equation}

For this we use the definition of $ad_\eta$ with the Lie brackets

\begin{equation}
	ad_{\eta_1}\eta_2 = \left[\hat{\eta}_1, \hat{\eta}_2 \right]^\vee
\end{equation}

Now as $\left[Ad_g \hat{\eta}_1, Ad_g  \hat{\eta}_2 \right]^\vee = Ad_g  \left[\hat{\eta}_1, \hat{\eta}_2 \right]^\vee$
we have that 

\begin{equation}
\begin{aligned}
	\left[Ad_g \hat{\eta}_1, Ad_g  \hat{\eta}_2 \right]^\vee &= ad_{Ad_g\eta_1} Ad_g \eta_2 \\
	Ad_g  \left[\hat{\eta}_1, \hat{\eta}_2 \right]^\vee		&= Ad_g ad_{\eta_1}\eta_2
\end{aligned}
\end{equation}

Thus we have proven that $Ad_g ad_{\eta_1}\eta_2 = ad_{Ad_g\eta_1} Ad_g \eta_2$. $\therefore$


We can use the previus Lemma 1 in order Equation \eqref{eq:twist derivative 2} having $Ad_{^jg_k} \eta_k$ as $\eta_2$ and $ad_{Ad_{^kg_j} {^j {\eta}}_{j/k}} Ad_{^kg_j} $ as $ad_{Ad_g\eta_1} Ad_g$ with $\eta_1 = {^j {\eta}}_{j/k} = {\eta}_{j/k}$. We thus have that $Ad_g ad_{\eta_1}\eta_2 = Ad_{^kg_j} ad_{{\eta}_{j/k}} Ad_{^jg_k} \eta_k$ and this can be used in the following expression

\begin{equation}\label{eq:twist derivative 3}
\begin{aligned}
	\dot{\eta}_j 	&= Ad_{^jg_k} \dot{\eta}_k -  Ad_{^jg_k} \left[ ad_{Ad_{^kg_j} {^j {\eta}}_{j/k}} Ad_{^kg_j}  Ad_{^jg_k} \eta_k \right] + \dot{\eta}_{j/k} \\
					&= Ad_{^jg_k} \dot{\eta}_k - Ad_{^jg_k} \left[ Ad_{^kg_j} ad_{{\eta}_{j/k}} Ad_{^jg_k} \eta_k \right] + \dot{\eta}_{j/k} \\
					&= Ad_{^jg_k} \dot{\eta}_k - ad_{{\eta}_{j/k}} Ad_{^jg_k} \eta_k + \dot{\eta}_{j/k} 
\end{aligned}
\end{equation}

Using the defintion of twist in Equation \eqref{eq: twist definition} we can obtain the relation $Ad_{^jg_k} \eta_k = \eta_j - \eta_{j/k}$. Using this last relation in Equation \eqref{eq:twist derivative 3} we obtain :

\begin{equation}\label{eq:twist derivative 4}
	\dot{\eta}_j 	= Ad_{^jg_k} \dot{\eta}_k - ad_{{\eta}_{j/k}} \left( \eta_j - \eta_{j/k} \right) + \dot{\eta}_{j/k} \\
\end{equation}

\paragraph{Lemma 2}
We want to proove that $ad_{\eta_i} \eta_i = 0$. If we take the defintion with Lie brackets be obtain : $ad_{\eta_i} \eta_i = \left[ \eta_i, \eta_i\right] = \eta_i \eta_i - \eta_i \eta_i = 0$ $\therefore$



We can use the lemma 2 to obtain the following : 

\begin{equation}\label{eq:twist derivative 5}
	\dot{\eta}_j 	= Ad_{^jg_k} \dot{\eta}_k - ad_{{\eta}_{j/k}}  \eta_j + \dot{\eta}_{j/k} \\
\end{equation}


knowing that $ad_{\eta_1} \eta_2 = - ad_{\eta_2} \eta_1$ We obtain the final formulation

\begin{equation}\label{eq:twist derivative final}
	\dot{\eta}_j 	= Ad_{^jg_k} \dot{\eta}_k + ad_{{\eta}_j}  \eta_{j/k} + \dot{\eta}_{j/k} \\
\end{equation}





\subsection{Delta dot Ad g}

Starting from the realation 
\begin{equation}
	\dot{g} = g \hat{\eta}
\end{equation}

We can obtain the variational part 

\begin{equation}
	\Delta g = g \Delta \hat{\zeta}
\end{equation}
 And similary extend this to the Adjoint operator \color{red}\textbf{proof required!}\color{black}

\begin{equation}\label{eq: Delta Ad g}
	\Delta Ad_g = Ad_g ad_{\Delta \zeta}
\end{equation}


Time derivating this equation we obtain :

\begin{equation}
	\Delta \dot{Ad}_g = \dot{Ad}_g ad_{\Delta \zeta} + Ad_g  ad_{\Delta \dot{\zeta} }
\end{equation}

Where using the definition in Equation \eqref{eq:derivative Ad time} we obtain 

\begin{equation}\label{eq: Delta dot Ad g}
	\Delta \dot{Ad}_g = Ad_g ad_\eta ad_{\Delta \zeta} + Ad_g  ad_{\Delta \dot{\zeta} }
\end{equation}

We need now to investigate the term $ad_{\Delta \dot{\zeta} }$. Using the definition $\Delta \dot{\zeta} = \Delta \eta - ad_\eta \Delta \zeta$ \color{red}\textbf{proof required!}\color{black} we obtain the following formulation 

\begin{equation}\label{eq: definition ad Delta dot zeta}
	ad_{\Delta \dot{\zeta}} = ad_{(\Delta \eta - ad_\eta \Delta \zeta)}
\end{equation}

Substituing Equation \eqref{eq: definition ad Delta dot zeta} into Equation \eqref{eq: Delta dot Ad g} we obtain the following expression

\begin{equation}\label{eq: Delta dot Ad g final}
\begin{aligned}
	\Delta Ad_g 	&= Ad_g ad_\eta ad_{\Delta \zeta} + Ad_g ad_{\Delta \eta} - Ad_g ad_{ad_\eta \Delta \zeta} \\
					&= Ad_g \left( ad_\eta ad_{\Delta \zeta} +  ad_{\Delta \eta} -  ad_{ad_\eta \Delta \zeta} \right)
\end{aligned}
\end{equation}



\subsection{Tangent Kinematics}

Here we want to derivate the formulation for the tangent kinematics of a Cosserat rod.

\subsubsection{Delta eta}

Knowing, from the analogy with the twist (Equation \eqref{eq: twist definition}), that we can express the variation on the pose for a body $j$ knowing the one of a body $k$, with $j>k$, and their relative pose variation $\Delta \zeta_{j/k}$ as follows.
\begin{equation}
	\Delta \zeta _ j = Ad_{^j g _k} \Delta \zeta _k + \Delta \zeta _ {j/k}
\end{equation}


Now for what concerns the twist, taking the variation of Equation \eqref{eq: twist definition} we obtain.
\begin{equation}
\begin{aligned}
	\Delta \eta_j 	&= \Delta \left( Ad_{^j g_k} \eta_k \right) + \Delta \eta_{j/k} \\
					&= \Delta  Ad_{^j g_k} \eta_k + Ad_{^j g_k} \Delta \eta_k  + \Delta \eta_{j/k} \\
\end{aligned}
\end{equation}

Using the expression in Equation \eqref{eq: Delta Ad g} we obtain 
\begin{equation}\label{eq: delta eta 1}
\begin{aligned}
	\Delta \eta_j 	&= Ad_{^j g_k} ad_{\Delta \zeta_{k/j}} \eta_k + Ad_{^j g_k} \Delta \eta_k  + \Delta \eta_{j/k} \\
\end{aligned}
\end{equation}


We now need to find a link between $\Delta \zeta_{k/j}$ and $\Delta \zeta_{j/k}$ as it is the latter to be available in the Newton-Euler algorithm.
For this we process as we did before : taking the variation of the identity ${^j g _k } {^k g _j } = \mathbb{1}$ we obtain the following expression.

\begin{equation}\label{eq: Delta equality poses}
\begin{aligned}
	\Delta {^j g _k } {^k g _j }	&= - {^j g _k } \Delta{^k g _j } \\
									&= - \Delta \zeta _{j/k}
\end{aligned}
\end{equation}

We can add the identity ${^j g _k } {^k g _j }$ on the left side of Equation \eqref{eq: Delta equality poses} to obtain 

\begin{equation}\label{eq: Delta equality poses 2 }
\begin{aligned}
	{^j g _k } {^k g _j } \Delta {^j g _k } {^k g _j }	&= - \Delta \zeta _{j/k}
\end{aligned}
\end{equation}

Now reminding the defintion of $\Delta \zeta $ from Equation \eqref{eq: Delta zeta} we obtain 

\begin{equation}\label{eq: Delta equality poses 3 }
\begin{aligned}
	{^j g _k } \Delta \hat{\zeta} _ {k/j} {^k g _j }	&= - \Delta \zeta _{j/k}
\end{aligned}
\end{equation}

We now have that $\left[ g \Delta \hat{\zeta} g \inv \right] = Ad_g \Delta \zeta$ so we obtain

\begin{equation}\label{eq: Delta equality poses 4 }
\begin{aligned}
	Ad_{^j g _k } \Delta \hat{\zeta} _ {k/j}	&= - \Delta \zeta _{j/k} \\
	\Delta \hat{\zeta} _ {k/j}	&= - Ad_{^j g _k }  \inv \Delta \zeta _{j/k} \\
								&= - Ad_{^j g _k \inv}   \Delta \zeta _{j/k} \\
								&= - Ad_{^k g _j}   \Delta \zeta _{j/k} \\
\end{aligned}
\end{equation}

Using Equation \eqref{eq: Delta equality poses 4 } into Equation \eqref{eq: delta eta 1} we obtain the following expression.

\begin{equation}\label{eq: delta eta 2}
	\Delta \eta_j 	= Ad_{^j g_k} ad_{- Ad_{^k g _j}   \Delta \zeta _{j/k}} \eta_k + Ad_{^j g_k} \Delta \eta_k  + \Delta \eta_{j/k}
\end{equation}

We can now introduce the identity $Ad_{^k g _j}Ad_{^j g _k}$ to obtain

\begin{equation}\label{eq: delta eta 3}
	\Delta \eta_j 	= - Ad_{^j g_k} ad_{Ad_{^k g _j}   \Delta \zeta _{j/k}} Ad_{^k g _j}Ad_{^j g _k}  \eta_k + Ad_{^j g_k} \Delta \eta_k  + \Delta \eta_{j/k}
\end{equation}

using the Lemma 1 described before, we have that $ad_{Ad_{^k g _j}   \Delta \zeta _{j/k}} Ad_{^k g _j} Ad_{^j g _k}  \eta_k = Ad_{^k g _j} ad_{\Delta \zeta _{j/k}} Ad_{^j g _k}  \eta_k$ thus we obtain


\begin{equation}\label{eq: delta eta 4}
\begin{aligned}
	\Delta \eta_j 	&= - Ad_{^j g_k} Ad_{^k g _j} ad_{\Delta \zeta _{j/k}} Ad_{^j g _k}  \eta_k + Ad_{^j g_k} \Delta \eta_k  + \Delta \eta_{j/k} \\
					&= Ad_{^j g_k} \Delta \eta_k - ad_{\Delta \zeta _{j/k}} Ad_{^j g _k}  \eta_k +  \Delta \eta_{j/k} 
\end{aligned}
\end{equation}

Now reminding that from Equation \eqref{eq: twist definition} $Ad_{^j g _k}  \eta_k = \eta_j - \eta_{j/k}$. using this equivalence in Equation \eqref{eq: delta eta 4} and applying directly the Lemma 2 we obtain the final formulation.

\begin{equation}\label{eq: delta eta 4}
	\Delta \eta_j 	= Ad_{^j g_k} \Delta \eta_k - ad_{Ad_{^j g _k}  \eta_k} \Delta \zeta _{j/k} +  \Delta \eta_{j/k} 
\end{equation}


\subsubsection{Delta dot eta}

Starting from the definition of the derivative of the local twist 


\begin{equation}
	\dot{\eta} = \Adjgk \doteta_k + ad_{\eta_j} \etajk + \dot{\eta}_{j/k}
\end{equation}

If we take the variation of this equation we obtain

\begin{equation}
	\Delta \dot{\eta} = \Adjgk \Delta \doteta_k + \Delta \Adjgk \doteta_k +  ad_{\eta_j} \Delta \etajk    +    ad_{\Delta \eta_j} \etajk    + \Delta \dot{\eta}_{j/k}
\end{equation}

Now we remind that $\Delta \Adjgk \doteta_k = \Adjgk{} ad_{\Delta \zeta_{k/j}} \doteta _k$ \color{red}\textbf{proof required!}\color{black}


\bibliography{bibliography}
\end{document}
